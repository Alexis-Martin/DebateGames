\documentclass{article}

\usepackage{graphicx}
\usepackage{subfigure}
\usepackage[utf8x]{luainputenc}
\usepackage{tikz}
\usetikzlibrary{graphdrawing,graphs}
\usegdlibrary{layered}

\begin{document}

\begin{figure}
\centering

\subfigure[Préférences du joueurs 1]{
\begin{tikzpicture}[>=stealth]
\graph [ layered layout, nodes = {scale=0.75, align=center} ] {
"a1\\ (0,1)" -> "q\\ (0,0)\\lm = 0.9998";
"a2\\ (1,0)" -> "a1\\ (0,1)";
"a3\\ (1,0)" -> "a1\\ (0,1)";
"a4\\ (0,1)" -> "a1\\ (0,1)";
"a5\\ (0,0)" -> "a3\\ (1,0)";
};
\end{tikzpicture}
}

\subfigure[Préférences du joueur 2]{
\begin{tikzpicture}[>=stealth]
\graph [ layered layout, nodes = {scale=0.75, align=center} ] {
"a1\\ (1,0)" -> "q\\ (0,0)\\lm = 0.99039";
"a2\\ (1,0)" -> "a1\\ (1,0)";
"a3\\ (0,1)" -> "a1\\ (1,0)";
"a4\\ (0,0)" -> "a1\\ (1,0)";
"a5\\ (0,1)" -> "a3\\ (0,1)";
};
\end{tikzpicture}
}


\subfigure[Résultat du jeu]{
\begin{tikzpicture}[>=stealth]
\graph [ layered layout, nodes = {scale=0.75, align=center} ] {
"a1\\ (0,1)\\ (1,0)" -> "q\\ (0,0)\\lm = 0.98725";
"a2\\ (1,0)\\ (0,0)" -> "a1\\ (0,1)\\ (1,0)";
"a3\\ (1,0)\\ (1,0)" -> "a1\\ (0,1)\\ (1,0)";
"a4\\ (1,0)\\ (0,1)" -> "a1\\ (0,1)\\ (1,0)";
"a5\\ (0,1)\\ (1,0)" -> "a3\\ (1,0)\\ (1,0)";
};
\end{tikzpicture}
}

\subfigure[Evolution de la valeur du jeu en fonction des coups]{
\includegraphics[scale=0.35]{not_in_range_tau_1.png}
}
\end{figure}

 \begin{figure}
       \subfigure[Liste des coups joués]{
       \begin{tabular}{|c|c|c|c|}
       \hline
       & joueur 1 & joueur 2 & general \\
       \hline
       tour 1 & \_ & \_ & 0.90625 \\
       \hline
       tour 2 & a1 dislike & \_ & 0.98125 \\
       \hline
       tour 3 & \_ & a5 dislike & 0.98625 \\
       \hline
       tour 4 & a4 like & \_ & 0.99725 \\
       \hline
       tour 5 & \_ & a1 like & 0.98625 \\
       \hline
       tour 6 & a2 like & \_ & 0.99725 \\
       \hline
       tour 7 & \_ & a4 dislike & 0.98625 \\
       \hline
       tour 8 & a3 like & \_ & 0.99525 \\
       \hline
       tour 9 & \_ & a5 annule & 0.98625 \\
       \hline
       tour 10 & a5 dislike & \_ & 0.99525 \\
       \hline
       tour 11 & \_ & a5 like & 0.98625 \\
       \hline
       tour 12 & pass & \_ & 0.98625 \\
       \hline
       tour 13 & \_ & a3 like & 0.98725 \\
       \hline
       tour 14 & pass & \_ & 0.98725 \\
       \hline
       tour 15 & \_ & pass & 0.98725 \\
       \hline
       \end{tabular}
       }
       \caption{Partie jouée avec la fonction $\tau_1$}
     \end{figure}
\end{document}
