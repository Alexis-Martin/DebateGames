\documentclass{article}

\usepackage{graphicx}
\usepackage{tikz}
\usetikzlibrary{graphdrawing,graphs}
\usegdlibrary{layered}
%Becareful if you use package fontenc, it might be raise an error. If it does, you have to remove it and use \usepackage[utf8x]{luainputenc} in place of \usepackage[utf8]{inputenc}

\begin{document}
\begin{figure}
\centering
\begin{tikzpicture}[>=stealth]
\graph [ layered layout, nodes = {scale=0.75, align=center} ] {
"a1\\ (0,1)" -> "q\\ (0,0)\\lm = 0.9039208";
"a2\\ (0,1)" -> "q\\ (0,0)\\lm = 0.9039208";
"a3\\ (0,1)" -> "q\\ (0,0)\\lm = 0.9039208";
"a5\\ (0,1)" -> "q\\ (0,0)\\lm = 0.9039208";
"a5\\ (0,1)" -> "q\\ (0,0)\\lm = 0.9039208";
"a4\\ (0,1)" -> "q\\ (0,0)\\lm = 0.9039208";
"a6\\ (0,1)"
};
\end{tikzpicture}
\caption{Joueur 3}
\end{figure}

\begin{figure}
\centering
\begin{tikzpicture}[>=stealth]
\graph [ layered layout, nodes = {scale=0.75, align=center} ] {
"a1\\ (0,1)" -> "q\\ (0,0)\\lm = 0.9039208";
"a2\\ (0,1)" -> "q\\ (0,0)\\lm = 0.9039208";
"a3\\ (0,1)" -> "q\\ (0,0)\\lm = 0.9039208";
"a5\\ (0,1)" -> "q\\ (0,0)\\lm = 0.9039208";
"a5\\ (0,1)" -> "q\\ (0,0)\\lm = 0.9039208";
"a4\\ (0,1)" -> "q\\ (0,0)\\lm = 0.9039208";
"a6\\ (0,1)"
};
\end{tikzpicture}
\caption{Joueur 2}
\end{figure}

\begin{figure}
\centering
\begin{tikzpicture}[>=stealth]
\graph [ layered layout, nodes = {scale=0.75, align=center} ] {
"a1\\ (1,0)" -> "q\\ (0,0)\\lm = 0.0";
"a2\\ (1,0)" -> "q\\ (0,0)\\lm = 0.0";
"a3\\ (1,0)" -> "q\\ (0,0)\\lm = 0.0";
"a5\\ (1,0)" -> "q\\ (0,0)\\lm = 0.0";
"a5\\ (1,0)" -> "q\\ (0,0)\\lm = 0.0";
"a4\\ (1,0)" -> "q\\ (0,0)\\lm = 0.0";
"a6\\ (1,0)"
};
\end{tikzpicture}
\caption{Joueur 1}
\end{figure}

\begin{figure}
\centering
\begin{tikzpicture}[>=stealth]
\graph [ layered layout, nodes = {scale=0.75, align=center} ] {
"a1\\ (1,0)\\ (0,1)\\ (0,1)" -> "q\\ (0,0)\\lm = 0.39154254";
"a2\\ (1,0)\\ (0,1)\\ (0,1)" -> "q\\ (0,0)\\lm = 0.39154254";
"a3\\ (1,0)\\ (0,1)\\ (0,1)" -> "q\\ (0,0)\\lm = 0.39154254";
"a5\\ (1,0)\\ (0,1)\\ (0,1)" -> "q\\ (0,0)\\lm = 0.39154254";
"a5\\ (1,0)\\ (0,1)\\ (0,1)" -> "q\\ (0,0)\\lm = 0.39154254";
"a4\\ (1,0)\\ (0,1)\\ (0,1)" -> "q\\ (0,0)\\lm = 0.39154254";
"a6\\ (0,0)"
};
\end{tikzpicture}
\caption{general}
\end{figure}
\begin{tabular}{|c|c|c|c|c|}
\hline
& Joueur 1 & Joueur 2 & Joueur 3 & general \\
\hline
tour 1 & \_ & \_ & \_ & 0.03125 \\
\hline
tour 2 & a5 like & \_ & \_ & 0.01068735 \\
\hline
tour 3 & \_ & a5 dislike & \_ & 0.03125 \\
\hline
tour 4 & \_ & \_ & a5 dislike & 0.05181265 \\
\hline
tour 5 & a4 like & \_ & \_ & 0.01771968 \\
\hline
tour 6 & \_ & a4 dislike & \_ & 0.05181265 \\
\hline
tour 7 & \_ & \_ & a4 dislike & 0.08590562 \\
\hline
tour 8 & a2 like & \_ & \_ & 0.02937931 \\
\hline
tour 9 & \_ & a2 dislike & \_ & 0.08590562 \\
\hline
tour 10 & \_ & \_ & a2 dislike & 0.14243194 \\
\hline
tour 11 & a3 like & \_ & \_ & 0.04871104 \\
\hline
tour 12 & \_ & a3 dislike & \_ & 0.14243194 \\
\hline
tour 13 & \_ & \_ & a3 dislike & 0.23615284 \\
\hline
tour 14 & a1 like & \_ & \_ & 0.08076313 \\
\hline
tour 15 & \_ & a1 dislike & \_ & 0.23615284 \\
\hline
tour 16 & \_ & \_ & a1 dislike & 0.39154254 \\
\hline
\end{tabular}
\end{document}