\documentclass{beamer}

\usepackage[frenchb]{babel}
\usepackage[utf8x]{luainputenc}
\usepackage{graphicx}
\usepackage{tikz}
\usetikzlibrary{graphdrawing,graphs}
\usegdlibrary{layered}

% There are many different themes available for Beamer. A comprehensive
% list with examples is given here:
% http://deic.uab.es/~iblanes/beamer_gallery/index_by_theme.html
% You can uncomment the themes below if you would like to use a different
% one:
%\usetheme{AnnArbor}
%\usetheme{Antibes}
%\usetheme{Bergen}
%\usetheme{Berkeley}
%\usetheme{Berlin}
%\usetheme{Boadilla}
%\usetheme{boxes}
%\usetheme{CambridgeUS}
%\usetheme{Copenhagen}
%\usetheme{Darmstadt}
\usetheme{default}
%\usetheme{Frankfurt}
%\usetheme{Goettingen}
%\usetheme{Hannover}
%\usetheme{Ilmenau}
%\usetheme{JuanLesPins}
%\usetheme{Luebeck}
%\usetheme{Madrid}
%\usetheme{Malmoe}
%\usetheme{Marburg}
%\usetheme{Montpellier}
%\usetheme{PaloAlto}
%\usetheme{Pittsburgh}
%\usetheme{Rochester}
%\usetheme{Singapore}
%\usetheme{Szeged}
%\usetheme{Warsaw}

\title{Les résultats}

\begin{document}

  \begin{frame}
    \titlepage
  \end{frame}

  \begin{frame}{Better response}
    Tous les tests indiquent qu'en better responses on converge.

    Par contre c'est plus long avant d'atteindre un équilibre.

    \begin{tabular}{cc}
      \includegraphics[scale=0.25]{/home/talkie/Documents/Stage/DebateGames/tests/27_06_15_27_round_robin_1/2_players/10_vertices/best_8_tau_1.png} &
      \includegraphics[scale=0.25]{/home/talkie/Documents/Stage/DebateGames/tests/27_06_15_27_round_robin_1/2_players/10_vertices/better_8_tau_1.png}
    \end{tabular}

    On remarque qu'en better response on arrive pas forcément sur le point de départ.
  \end{frame}

  \begin{frame}{Analyse approfondie d'un jeu à 2 joueurs}
    \begin{minipage}{.5\linewidth}
    \begin{figure}
    \centering
    \begin{tikzpicture}[>=stealth]
    \graph [ layered layout, nodes = {scale=0.60, align=center} ] {
    "a1\\ (0,0)" -> "q\\ (0,0)\\lm = 0.0102";
    "a2\\ (0,1)" -> "a1\\ (0,0)";
    "a3\\ (1,0)" -> "q\\ (0,0)\\lm = 0.0102";
    };
    \end{tikzpicture}
    \caption{joueur 1}
    \end{figure}

    \begin{figure}
    \centering
    \begin{tikzpicture}[>=stealth]
    \graph [ layered layout, nodes = {scale=0.60, align=center} ] {
    "a1\\ (0,0)" -> "q\\ (0,0)\\lm = 0.375";
    "a2\\ (0,0)" -> "a1\\ (0,0)";
    "a3\\ (0,0)" -> "q\\ (0,0)\\lm = 0.375";
    };
    \end{tikzpicture}
    \caption{initialisation}
    \end{figure}

    \end{minipage}\hfill
    \begin{minipage}{.5\linewidth}
      \begin{figure}
      \centering
      \begin{tikzpicture}[>=stealth]
      \graph [ layered layout, nodes = {scale=0.60, align=center} ] {
      "a1\\ (1,0)" -> "q\\ (0,0)\\lm = 0.4902";
      "a2\\ (1,0)" -> "a1\\ (1,0)";
      "a3\\ (0,0)" -> "q\\ (0,0)\\lm = 0.4902";
      };
      \end{tikzpicture}
      \caption{joueur 2}
      \end{figure}

    \begin{figure}
    \centering
    \begin{tikzpicture}[>=stealth]
    \graph [ layered layout, nodes = {scale=0.60, align=center} ] {
    "a1\\ (1,0)\\ (0,1)" -> "q\\ (0,0)\\lm = 0.375";
    "a2\\ (0,1)\\ (1,0)" -> "a1\\ (1,0)\\ (0,1)";
    "a3\\ (1,0)\\ (0,1)" -> "q\\ (0,0)\\lm = 0.375";
    };
    \end{tikzpicture}
    \caption{équilibre}
    \end{figure}
    \end{minipage}
  \end{frame}

  \begin{frame}{Résultats}
    \begin{itemize}
      \item Quelque soit la valeur des deux joueurs, le jeu converge (même en better response).
      \item il y a toujours une valeur du jeu entre les valeurs possibles des joueurs (A TESTER)
      \item Il y a toujours un point d'équilibre entre la valeur des deux joueurs. (A TESTER)
    \end{itemize}

    \begin{itemize}
      \item Est-ce qu'un cycle peut arriver en better response et/ou random? (A TESTER)
    \end{itemize}
  \end{frame}

  \begin{frame}{Fonctions potentielles}
    \begin{itemize}
      \item A un joueur il y a une fonction potentielle évidente.
      \item A deux joueurs (ou plus) ce n'est plus vrai.
      \item il faut trouver une fonction qui croît (ou décroît) malgré le fait que $v_f$ peut être égale à $v_0$.(Il faut donc inclure les votes des arguments ?)
      \item Serait-il possible via une fonction de quantifier le nombre de configurations améliorantes?
    \end{itemize}


  \end{frame}
\end{document}
