\documentclass[12pt]{article}

\usepackage{amsmath}
\usepackage{amssymb}
\usepackage[]{geometry}
\usepackage[frenchb]{babel}
\usepackage[utf8x]{luainputenc}
\usepackage{mathtools}
\usepackage{amsthm}
\usepackage{color}
\usepackage{graphicx}
\usepackage{tikz}
\usetikzlibrary{graphdrawing,graphs}
\usegdlibrary{layered}

\title{Compte rendu de la semaine}
\author{Alexis Martin}

\newtheoremstyle{not}
{\topsep}
{\topsep}
{}
{}
{\bfseries}
{.}
{\newline}
{}

\newtheoremstyle{defi}
{\topsep}
{\topsep}
{\itshape}
{}
{\bfseries}
{.}
{\newline}
{}

\newtheoremstyle{prob}
{\topsep}
{\topsep}
{}
{}
{\bfseries}
{.}
{\newline}
{}

\newtheorem{proposition}{Proposition}[section]
\theoremstyle{defi}
\newtheorem{definition}{Définition}[section]
\theoremstyle{not}
\newtheorem{notation}{Notation}[section]
\theoremstyle{prob}
\newtheorem{problem}{Problème}[section]
\newtheorem{solution}{Solution}[problem]



\begin{document}
  \maketitle

  \section{Légende}
    \color{black}
      Ce qui est ancien et qui reste.

    \color{blue}
      Ce qui est a été ajouté cette semaine.

    \color{red}
      Ce qui est amené à disparaitre dans les semaines suivantes.

      P.S.: Le fichier Tex est sur Git, on aura donc toujours la possibilité de retrouver ce qui a été éffacé.

  \color{black}
  \section{Rappels}
    \subsection{Définitions}
      \begin{definition}[Graphe d'Argumentation]

        $F = \langle \mathcal{A}, \mathcal{R}, V \rangle$ avec :
        \begin{itemize}
          \item $\mathcal{A}$ : L'ensemble des arguments.
          \item $\mathcal{R} \subseteq \mathcal{A} \times \mathcal{A}$ : Une relation binaire entre arguments tels que : $(a,b) \in \mathcal{R}$ si $a$ "attaque" $b$.
          \item $V: \mathcal{A} \rightarrow \mathbb{N} \times \mathbb{N} :$ La fonction qui associe, à chaque argument, le nombre de "like" et de "dislike", $V(a) = (v^+, v^-)$ signifie que $a$ à $v^+$ "like" et $v^-$ "dislike".
        \end{itemize}
      \end{definition}

      \begin{notation}
        On note $Att(a)$ l'ensemble des arguments directs qui attaquent $a$ :
        $Att(a) = \{b\ |\ (b, a)\in \mathcal{R}\}$
      \end{notation}

      \begin{definition}
        \label{ref:def_tau}
        Soit $\tau_1$ la fonction défini par :
        $$
          \begin{array}{rclc}
            \tau_1 :  & \mathbb{N} \times \mathbb{N} & \longrightarrow & [0, 1] \\
            & (v^+,v^-) & \longmapsto & \frac{1}{2} \cdot \left(1 + sgn(v^+ - v^-) \cdot \left(1 - e^{\frac{-|v^+ - v^-|}{\varepsilon}}\right)\right)\\
          \end{array}
        $$

        où
        $sgn(x) = \left\{
          \begin{array}{ll}
            -1  & \mbox{si } x < 0 \\
            1 & \mbox{sinon} \\
          \end{array}
        \right.$
        et $\varepsilon = - \frac{\#joueurs}{ln(0.04)}$
      \end{definition}

      \paragraph{remarques/variantes}
      \label{ref:rem_def_tau}
        \begin{enumerate}
          \item On peut noter que cette fonction est la fonction de répartition de la première loi de Laplace.
          \item Avec cette définition on n'obtient pas une valeur de 1 (resp. $0$) sur l'argument $a$ si tous les agents "like" (resp. "dislikes"). On peut donc raffiner cette fonction et forcer ces cas limites :
            $$
              \begin{array}{rclc}
                \tau_2 :  & \mathbb{N} \times \mathbb{N} & \longrightarrow & [0, 1] \\
                & (\#joueurs, 0) & \longmapsto & 1 \\
                & (0, \#joueurs) & \longmapsto & 0 \\
                & (v^+,v^-) & \longmapsto & \frac{1}{2} \cdot \left(1 + sgn(v^+ - v^-) \cdot \left(1 - e^{\frac{-|v^+ - v^-|}{\varepsilon}}\right)\right)\\
              \end{array}
            $$\\
        \end{enumerate}
\color{blue}
      \begin{proposition}
        Les fonctions $\tau$ sont croissantes selon $v^+$ et décroissantes selon $v^-$.
      \end{proposition}
      \begin{proof}
        On va montrer la croissance (resp. décroissance) selon $v^+$ (resp. $v^-$) pour la fonction $\tau_1$ et cette preuve sera suffisante pour se convaincre pour $tau_2$.

        Commençons par remarquer que :
        $$tau_1(v^+, v^-) = \left\{
          \begin{array}{ll}
            \frac{1}{2} \cdot \exp(\frac{v^+ - v^-}{\varepsilon})  & \mbox{si } v^+ - v^- < 0 \\
            1 - \frac{1}{2} \cdot \exp(\frac{v^- - v^+}{\varepsilon}) & \mbox{sinon} \\
          \end{array}
        \right.$$

        Soit $v^- \in \mathcal{N}$ et $v_1^+, v_2^+ \in \mathcal{N}$ tel que $v_1^+ < v_2^+$. Montrons que $tau_1(v_1^+, v^-) < tau_1(v_2^+, v^-)$

        3 cas possibles :
        \begin{itemize}
          \item si $v_1^+ - v^- < 0$ et $v_2^+ - v^- < 0$
          $$tau_1(v_1^+, v^-) - tau_1(v_2^+, v^-) = \frac{1}{2} \cdot \exp(\frac{- v^-}{\varepsilon})(\exp(\frac{v_1^+}{\varepsilon}) - \exp(\frac{v_2^+}{\varepsilon}))$$

          $\exp$ est une fonction croissante donc $tau_1(v_1^+, v^-) < tau_1(v_2^+, v^-)$

          \item si $v_1^+ - v^- < 0$ et $v_2^+ - v^- \geq 0$
          $$tau_1(v_1^+, v^-) - tau_1(v_2^+, v^-) = \frac{1}{2} \cdot \exp(\frac{v_1^+ - v^-}{\varepsilon}) - 1 + \frac{1}{2} \cdot \exp(\frac{v^- - v_2^+}{\varepsilon})$$

          En remarquant que $\exp(\frac{v_1^+ - v^-}{\varepsilon}) < 1$ et que $\exp(\frac{v^- - v_2^+}{\varepsilon}) \leq 1$ on en conclut que $tau_1(v_1^+, v^-) < tau_1(v_2^+, v^-)$

          \item si $v_1^+ - v^- \geq 0$ et $v_2^+ - v^- \geq 0$
          $$tau_1(v_1^+, v^-) - tau_1(v_2^+, v^-) = \frac{1}{2} \cdot \exp(\frac{v^-}{\varepsilon})(\exp(\frac{- v_2^+}{\varepsilon}) - \exp(\frac{- v_1^+}{\varepsilon}))$$

          $\exp(\frac{- v_2^+}{\varepsilon}) < \exp(\frac{- v_1^+}{\varepsilon})$ donc $tau_1(v_1^+, v^-) < tau_1(v_2^+, v^-)$
        \end{itemize}

        La preuve est la même pour la décroissance selon $v^-$.
      \end{proof}
\color{black}
      On peut ensuite définir une sémantique d'argumentation.
      \begin{definition}[Laplace Product Semantics]
        $\mathcal{S}_1$ (resp $\mathcal{S}_2$) $= \langle [0, 1], \tau_\varepsilon, \curlywedge, \curlyvee, \neg  \rangle$ est la sémantique tel que :
        \begin{enumerate}
          \item $x_1 \curlywedge x_2 = x_1 \cdot x_2$
          \item $x_1 \curlyvee x_2 = x_1 + x_2 - x_1 \cdot x_2$
          \item $\neg x_1 = 1 - x_1$
          \item $\tau_1$ (resp. $\tau_2$) la fonction défini précédement.
        \end{enumerate}
      \end{definition}

      On notera par la suite $\mathcal{S}$ lorsque $\mathcal{S}_1$ ou $\mathcal{S}_2$ peuvent être utilisée indifférement.
      De même on notera $\tau$ pour désigner $\tau_1$ ou $\tau_2$.

      \begin{definition}[modèle social]
        Soit $F= \langle \mathcal{A}, \mathcal{R}, V \rangle$ un graphe d'argumentation et $\mathcal{S} = \langle [0, 1], \tau, \curlywedge, \curlyvee, \neg  \rangle$.
        La fonction $LM : \mathcal{A} \rightarrow [0, 1]$ est un $\mathcal{S}$-Model pour $F$ si :

        $LM(a) = \tau(a) \curlywedge \neg$ {\Large $\curlyvee$} $\{LM(a_i)\ |\ a_i \in Att(a)\}$
      \end{definition}

  \section{Les modèles}
    Soit $F = \langle \mathcal{A}, \mathcal{R}, V \rangle$ un graphe d'argumentation, $\mathcal{J} = \{1, \ldots, n\}$ $n$ agents ($=$ joueurs) et $\mathcal{S}$ la sémantique utilisée.
    Chaque agent connait tout le graphe $F$, ils connaissent donc tous les arguments et la question.

    \paragraph{Initialisation}
      \begin{enumerate}
        \item Chaque agent $k$ analyse le débat, il "like" les arguments qu'il trouve recevables et "dislike" ceux qu'il juge inconsistant.
        Les arguments recevables auront donc 1 "like" et 0 "dislike", ceux inconsistant auront 0 "like" et 1 "dislike" et les autres 0 "like" et "dislike".
        \item En tenant compte de ces poids il calcule la valeur qu'il associe à la question. Cette valeur n'est pas connu des autres agents.
      \end{enumerate}
      \begin{definition}
        Soit $F_k = \langle \mathcal{A}, \mathcal{R}, V_k \rangle$ le graphe d'argumentation du joueur $k$.
        La fonction $V_k$ définie donc la préférence du joueur $k$ sur chacun des arguments de $\mathcal{A}$.
      \end{definition}
      \begin{definition}
        Soit $LM_k$ un $\mathcal{S}$-Model pour le graphe $F_k$.

        On appelle $LM_\mathcal{J}$ la fonction qui associe à chaque joueur la valeur qu'il associe à la question.

        $$
          \begin{array}{rclc}
            LM_\mathcal{J} :  & \mathcal{J} & \longrightarrow & [0, 1] \\
            & k & \longmapsto & LM_k(q)\\
          \end{array}
        $$
      \end{definition}


    \paragraph{Le jeu}(deux variantes possibles)
      \begin{description}
        \item[Variante 1] Les agents se regroupent autour d'une table où seul la question est exposée. Ils reconstruisent le débat en ajoutant les arguments les uns après les autres. Pour ce faire lorsqu'un agent vote ("like" ou "dislike") sur un argument non présent, l'argument s'ajoute à la représentation avec le vote.

        \item[Variante 2] Les agents se regroupent autour d'une table où une représentation du débat s'y trouve. Cette représentation est dépourvu de poids ("like" ou "dislike") et la valeur de la question dans cette représentation est affiché et connu de tous.
      \end{description}

    \paragraph{objectif}(deux objectifs différents possibles)
      \begin{description}
        \item[Objectif 1] L'objectif pour chacun des joueurs va être de faire tendre la valeur de la question de cette représentation vers sa valeur calculée lors de la phase d'initialisation.
        \item[Objectif 2] L'intervalle $[0, 1]$ est divisé en cluster. Un joueur appartient à un cluster si sa valeur appartient au cluster.

        L'objectif de chaque joueur est que la valeur finale du débat appartienne à son cluster. Les joueurs peuvent alors former des alliances.
      \end{description}


    \paragraph{Règles}(deux variantes possibles)
      \begin{description}
        \item[Variante 1] Chaque agent peut voter qu'une fois (k fois) par argument.

        \item[Variante 2] Les agents peuvent changer d'avis sur le vote qu'ils ont attribué à un argument et donc retirer leur vote ou l'inverser.
      \end{description}

    \paragraph{Dynamique\\}
      On considère dans ces modèles une dynamique de meilleure réponse. Round Robin.

    \paragraph{Les questions qu'on se pose}
      \begin{itemize}
        \item Il y a t-il convergence?
        \item Si tous les agents ont la même valeur à la fin de l'initialisation, il y a t-il convergence? Il existe alors un équilibre mais peut-on toujours l'atteindre? (Ce n'est pas la même question? Si on peut toujours atteindre l'équilibre alors ça converge, et si ça converge alors il y a un équilibre? Non?)
        \item Quel est le prix de l'anarchie dans le cas de la règle 1?
        \item Peut-on trouver une stratégie pour les agents?
      \end{itemize}



  \section{Problèmes liés aux modèles}
    \begin{problem}
      Dans notre cas, on considère que la question (notons la $q$) ne reçoit aucun vote (ce qui est plutôt intuitif). Mais si on prend cette définition pour la sémantique alors $LM(q) = 0$ quelques soit les attaquants car $\tau_\varepsilon(q) = 0$.
    \end{problem}

    \begin{solution}
      On a changé la définition de $\tau$ cf. définition \ref{ref:def_tau} page \pageref{ref:def_tau}
    \end{solution}

    \begin{problem}
      Un autre problème un peu similaire au précédent.
      D'après la définition de Leite et Martins un argument qui n'a pas de vote a un poids à 0 (i.e. $\tau_\varepsilon (a) = 0$) et un vote qui n'a que des votes négatif a également un poids à 0.

      Le problème dans l'initialisation (entre autre) est que "disliker" un argument (ne pas le trouver recevable du tout) ou ne pas avoir d'avis reviens au même, ce fait ce propage à tous les fils de ces arguments.

      Cela peut paraitre incohérent.
    \end{problem}

    \begin{solution}
      La nouvelle définition de $\tau$ (cf. définition \ref{ref:def_tau} page \pageref{ref:def_tau}) corrige également cette incohérence.
    \end{solution}


  \section{Tests et Exemples}
    \subsection{Le cas des Arbres}
      \subsubsection{Question: Peut-on trouver des arbres ou les agents n'ont pas les mêmes préférences mais la même valeur à la fin de l'initialisation?\newline}

        \paragraph{Si l'on la fonction $\tau_2$}

          \subparagraph{rappel} Cette fonction $\tau_2$ force les cas limites (c.f. remarque de la definition \ref{ref:def_tau}) \\

          Sur le débat représenté dans la figure \ref{fig:dif_pref} on remarque que le joueur 1 et le joueur 2 ont des avis totalement différents sur chacun des arguments. En revanche la valeur de la question est la même. A savoir : $LM_1(q) = LM_2(q) = 0.125$

          \begin{figure}
            \centering
            \begin{tabular}{ccc}
              \begin{tikzpicture}[>=stealth]
              \graph [ layered layout, nodes = {scale=0.75, align=center} ] {
              "a1\\ (0,0)" -> "q\\ (0,0)";
              "a2\\ (0,0)" -> "q\\ (0,0)";
              "a3\\ (0,0)" -> "a1\\ (0,0)";
              "a4\\ (0,0)" -> "a1\\ (0,0)";
              "a5\\ (0,0)" -> "a2\\ (0,0)";
              "a6\\ (0,0)" -> "a2\\ (0,0)";
              "a7\\ (0,0)" -> "a3\\ (0,0)";
              "a8\\ (0,0)" -> "a3\\ (0,0)";
              };
              \end{tikzpicture} &
              \begin{tikzpicture}[>=stealth]
              \graph [ layered layout, nodes = {scale=0.75, align=center} ] {
              "a1\\ (1,0)" -> "q\\ (0,0)";
              "a2\\ (0,1)" -> "q\\ (0,0)";
              "a3\\ (1,0)" -> "a1\\ (1,0)";
              "a4\\ (0,1)" -> "a1\\ (1,0)";
              "a5\\ (1,0)" -> "a2\\ (0,1)";
              "a6\\ (1,0)" -> "a2\\ (0,1)";
              "a7\\ (0,0)" -> "a3\\ (1,0)";
              "a8\\ (0,0)" -> "a3\\ (1,0)";
              };
              \end{tikzpicture} &
              \begin{tikzpicture}[>=stealth]
              \graph [ layered layout, nodes = {scale=0.75, align=center} ] {
              "a1\\ (0,0)" -> "q\\ (0,0)";
              "a2\\ (0,0)" -> "q\\ (0,0)";
              "a3\\ (0,1)" -> "a1\\ (0,0)";
              "a4\\ (0,1)" -> "a1\\ (0,0)";
              "a5\\ (0,1)" -> "a2\\ (0,0)";
              "a6\\ (0,1)" -> "a2\\ (0,0)";
              "a7\\ (1,0)" -> "a3\\ (0,1)";
              "a8\\ (1,0)" -> "a3\\ (0,1)";
              };
              \end{tikzpicture} \\
            \end{tabular}

            \caption{De gauche à droite: graphe général, préférences du joueur 1, préférences du joueur 2}
            \label{fig:dif_pref}
          \end{figure}

        \paragraph{Si l'on considère la fonction $\tau_1$\\}
          Je n'ai pas encore trouvé d'exemple qui possède cette propriété.
          Cela est dû à la définition de $\tau_1$, un argument "disliker" n'aura pas un poids égale à 0 et un argument "liker" n'aura pas un poids égale à 1.

      %TODO Modifier cette partie sur l'équilibre. En discuter avec Nicolas et Elise
      \subsubsection{Règle du jeu : ONESHOT Question: Si tous les agents ont la même valeur à la fin de l'initialisation, peut-on atteindre l'équilibre?}

        \paragraph{Si l'on considère la fonction $\tau_2$\\}

          Sur l'exemple précédent (c.f. figure \ref{fig:dif_pref}) on atteint l'équilibre.

          Sur l'exemple trivial de la figure \ref{fig:trivial} on atteint également l'équilibre.

          \begin{figure}
            \centering
            \begin{tabular}{cccc}
              \begin{tikzpicture}[>=stealth]
              \graph [ layered layout, nodes = {scale=0.75, align=center} ] {
              "a1\\ (0,0)" -> "q\\ (0,0)";
              "a2\\ (0,0)" -> "q\\ (0,0)";
              "a3\\ (0,0)" -> "q\\ (0,0)";
              };
              \end{tikzpicture} &

              \begin{tikzpicture}[>=stealth]
              \graph [ layered layout, nodes = {scale=0.75, align=center} ] {
              "a1\\ (1,0)" -> "q\\ (0,0)";
              "a2\\ (0,1)" -> "q\\ (0,0)";
              "a3\\ (0,1)" -> "q\\ (0,0)";
              };
              \end{tikzpicture} &

              \begin{tikzpicture}[>=stealth]
              \graph [ layered layout, nodes = {scale=0.75, align=center} ] {
              "a1\\ (0,1)" -> "q\\ (0,0)";
              "a2\\ (1,0)" -> "q\\ (0,0)";
              "a3\\ (0,1)" -> "q\\ (0,0)";
              };
              \end{tikzpicture} &

              \begin{tikzpicture}[>=stealth]
              \graph [ layered layout, nodes = {scale=0.75, align=center} ] {
              "a1\\ (0,1)" -> "q\\ (0,0)";
              "a2\\ (0,1)" -> "q\\ (0,0)";
              "a3\\ (1,0)" -> "q\\ (0,0)";
              };
              \end{tikzpicture} \\
            \end{tabular}

            \caption{De gauche à droite: graphe général, préférences du joueur 1, préférences du joueur 2, préférences du joueur 3}
            \label{fig:trivial}
          \end{figure}

          \subparagraph{Remarque\\}
            Sur cet exemple, il est intéressant de noter que $LM_1(q) = LM_2(q) = LM_3(q) = 0$.
            En effet $LM(q) = 0.5 \cdot (1 - \curlyvee \{LM(a) | a\in Att(q)\})$ et $\curlyvee \{LM(a) | a\in Att(q)\} = 1$


        \paragraph{Si l'on considère $\tau_1$\\}
          A priori, on n'atteint pas l'équilibre.

          Sur l'exemple de la figure \ref{fig:dif_pref} on atteint l'équilibre, en revanche sur la figure \ref{fig:trivial}, il semblerai qu'on ne l'atteigne pas ($LM_1(q) = LM_2(q) = LM_3(q) = 0.009604$ alors que $LM_g(q) = 0.00731$).

          \subparagraph{remarque\\} Je me demande si cela n'est pas dû à une précision machine car le système se stabilise dans une position ou un coup est très proche d'être meilleur.
            Je ferais le détail des calculs la semaine prochaine pour être sur.

      \subsubsection{Question: Notre définition de $\tau$ change t'elle radicalement les résultats obtenus sur les exemples de Leite et Martins?}

        \paragraph{Le premier exemple\\}

          \begin{figure}
            \centering
            \begin{tikzpicture}[>=stealth]
              \graph [ layered layout, nodes = {scale=0.75, align=center} ] {
              "a\\ (20,20)" -> "b\\ (20,20)";
              "b\\ (20,20)" -> "a\\ (20,20)";
              "c\\ (60,10)" -> "b\\ (20,20)";
              "e\\ (40,10)" -> "c\\ (60,10)";
              "d\\ (10,40)" -> "c\\ (60,10)";
              "e\\ (40,10)" -> "d\\ (10,40)";
              "d\\ (10,40)" -> "e\\ (40,10)";
              };
            \end{tikzpicture}
            \caption{Exemple de l'article Social Abstract Argumentation de Leite et Martins}
            \label{fig:LM_example}
          \end{figure}

          L'exemple de Leite et Martins est représenté figure \ref{fig:LM_example} et le tableau ci-dessous référence les valeurs de chacun des arguments trouvés par Leite et Martins ainsi que les notres.

          \begin{tabular}{|c|c|c|c|c|c|}
            \hline
                       & a    & b    & c    & d     & e \\
            \hline
            LM         & 0.37 & 0.25 & 0.19 & 0.05  & 0.76 \\
            \hline
            def $\tau$ & 0.36 & 0.28 & 0.13 & 0.018 & 0.86\\
            \hline
          \end{tabular}

          Il n'y a donc pas de différence fondamentale, les arguments sont toujours classés dans le même ordre.

        \paragraph{Le deuxième exemple\\}
          Cet exemple est tiré de l'article Extending Social Abstract Argumentation with Votes on Attacks.
          Il est donné figure \ref{fig:LM2_example}. Les résultats observés sont :

          \begin{figure}
            \centering
            \begin{tikzpicture}[>=stealth]
              \graph [ layered layout, nodes = {scale=0.75, align=center} ] {
              "a\\ (70,70)" -> "b\\ (70,70)";
              "b\\ (70,70)" -> "a\\ (70,70)";
              "c\\ (54,66)" -> "a\\ (70,70)";
              "e\\ (19,1)" -> "a\\ (70,70)";
              "d\\ (130,61)" -> "c\\ (54,66)";
              "d\\ (130,61)" -> "b\\ (70,70)";
              };
            \end{tikzpicture}
            \caption{Deuxième exemple de Leite et Martins}
            \label{fig:LM2_example}
          \end{figure}

          \begin{tabular}{|c|c|c|c|c|c|}
            \hline
                       & a    & b     & c     & d    & e \\
            \hline
            LM         & 0.02 & 0.16  & 0.14  & 0.68 & 0.95 \\
            \hline
            def $\tau$ & 0.16 & 0.066 & 0.063 & 0.84 & 0.63\\
            \hline
          \end{tabular}

          En revanche sur cet exemple les changements sont plus importants.

          Avec la défintion de Leite et Martins on a : $a \prec c \prec b \prec d \prec e$ alors qu'avec notre définition on a : $c \prec b \prec a \prec e \prec d$


  \section{Résultats}
    \subsection{Atteint-on l'équilibre avec la première définition de $\tau$ dans l'exemple de la figure \ref{fig:trivial}?}

      \paragraph{remarque :}
        Je viens de me rendre compte que j'ai peut être mal interprété le terme équilibre.
        J'entend par équilibre le fait d'atteindre la valeur exact souhaité par les 3 joueurs (dans le cadre de l'exemple de la figure \ref{fig:trivial}).
        C'est à dire 0.009604.

        La réponse est non, j'ai refait le calcul exact à la main et on atteint pas 0.009604 avec ce point de départ.\\

    En revanche cet objectif existe, si on prend :
    \begin{itemize}
      \item a1 : $likes = 3, dislikes = 0$
      \item a2 : $likes = 0, dislikes = 3$
      \item a3 : $likes = 0, dislikes = 3$
    \end{itemize}

    Alors $LM(q) = 0.009604$\\

    En changeant le nombre de vote au départ, par exemple si on prend :
    \begin{itemize}
      \item a1 : $likes = 1, dislikes = 2$
      \item a2 : $likes = 1, dislikes = 2$
      \item a3 : $likes = 1, dislikes = 2$
    \end{itemize}

    on atteint pas non plus l'équilibre, on reste bloqué comme avant à environ $0.00731$.
\color{red}
    \subsection{La convergence sur les arbres n'est pas garantie}
      Après avoir créé un générateur qui est capable de créer un jeu et d'exécuter la règle \texttt{mind\_changed} j'ai trouvé plusieurs exemples qui semblent contredire la convergence sur les arbres.

      Le plus étrange d'entre eux est un jeu à 3 joueurs avec 10 noeuds. Cet arbre est représenté figure \ref{fig:non_conv}

      \begin{figure}
        \centering
        \begin{tabular}{cc}
          \begin{tikzpicture}[>=stealth]
          \graph [ layered layout, nodes = {scale=0.75, align=center} ] {
          "a1\\ (0,0)" -> "q\\ (0,0)";
          "a2\\ (0,0)" -> "a1\\ (0,0)";
          "a3\\ (0,0)" -> "a2\\ (0,0)";
          "a4\\ (0,0)" -> "q\\ (0,0)";
          "a5\\ (0,0)" -> "a1\\ (0,0)";
          "a6\\ (0,0)" -> "a2\\ (0,0)";
          "a7\\ (0,0)" -> "q\\ (0,0)";
          "a8\\ (0,0)" -> "a1\\ (0,0)";
          "a9\\ (0,0)" -> "a6\\ (0,0)";
          };
          \end{tikzpicture} &

          \begin{tikzpicture}[>=stealth]
          \graph [ layered layout, nodes = {scale=0.75, align=center} ] {
          "a1\\ (0,0)" -> "q\\ (0,0)";
          "a2\\ (0,0)" -> "a1\\ (0,0)";
          "a3\\ (0,1)" -> "a2\\ (0,0)";
          "a4\\ (0,0)" -> "q\\ (0,0)";
          "a5\\ (1,0)" -> "a1\\ (0,0)";
          "a6\\ (1,0)" -> "a2\\ (0,0)";
          "a7\\ (0,0)" -> "q\\ (0,0)";
          "a8\\ (0,1)" -> "a1\\ (0,0)";
          "a9\\ (0,1)" -> "a6\\ (1,0)";
          };
          \end{tikzpicture} \\

          \begin{tikzpicture}[>=stealth]
          \graph [ layered layout, nodes = {scale=0.75, align=center} ] {
          "a1\\ (0,1)" -> "q\\ (0,0)";
          "a2\\ (1,0)" -> "a1\\ (0,1)";
          "a3\\ (0,0)" -> "a2\\ (1,0)";
          "a4\\ (0,0)" -> "q\\ (0,0)";
          "a5\\ (0,0)" -> "a1\\ (0,1)";
          "a6\\ (1,0)" -> "a2\\ (1,0)";
          "a7\\ (0,1)" -> "q\\ (0,0)";
          "a8\\ (0,1)" -> "a1\\ (0,1)";
          "a9\\ (0,0)" -> "a6\\ (1,0)";
          };
          \end{tikzpicture} &

          \begin{tikzpicture}[>=stealth]
          \graph [ layered layout, nodes = {scale=0.75, align=center} ] {
          "a1\\ (1,0)" -> "q\\ (0,0)";
          "a2\\ (0,0)" -> "a1\\ (1,0)";
          "a3\\ (0,1)" -> "a2\\ (0,0)";
          "a4\\ (0,1)" -> "q\\ (0,0)";
          "a5\\ (1,0)" -> "a1\\ (1,0)";
          "a6\\ (1,0)" -> "a2\\ (0,0)";
          "a7\\ (0,0)" -> "q\\ (0,0)";
          "a8\\ (0,1)" -> "a1\\ (1,0)";
          "a9\\ (0,1)" -> "a6\\ (1,0)";
          };
          \end{tikzpicture} \\
        \end{tabular}

        \caption{De gauche à droite: graphe général, préférences du joueur 1, préférences du joueur 2, préférences du joueur 3}
        \label{fig:non_conv}
      \end{figure}

      En effet sur ce graphe à partir d'un certain temps 1 joueur ne joue plus du tout et les 2 autres changent d'avis sur un seul argument (l'argument a6).

      Le tableau ci-dessous montre la valeur des préférences des trois joueurs au moment de la situation initiale puis les coups joués. La dernière colonne est la valeur du graphe général (le graphe du jeu) au fur et à mesure du temps :

      \begin{tabular}{|c|c|c|c|c|}
        \hline
                            & joueur 1   & joueur 2 & joueur 3    & general \\
        \hline
        situation initiale  & 0.125      & 0.25     & 0.25        & 0.1123 \\
        \hline
        départ de la boucle & $\cdots$   & $\cdots$ & $\cdots$    & 0.20688 \\
        \hline
        tour 1              & \_         & \_       & a6 like     & 0.20699 \\
        \hline
        tour 2              & a6 dislike & \_       & \_          & 0.20688 \\
        \hline
        tour 3              & \_         & \_       & a6 dislike  & 0.2070 \\
        \hline
        tour 4              & a6 like    & \_       & \_          & 0.20688 \\
        \hline
        tour 5              & \_         & \_       & a6 annule   & 0.20699 \\
        \hline
        tour 6              & a6 annule  & \_       & \_          & 0.20688 \\
        \hline
        tour 7              & \_         & \_       & a6 like     & 0.20699 \\
        \hline
        tour 8              & a6 dislike & \_       & \_          & 0.20688 \\
        \hline
      \end{tabular}

      le graphe au départ de la boucle est représenté figure \ref{fig:avant_boucle}

      \begin{figure}
      \centering
      \begin{tikzpicture}[>=stealth]
      \graph [ layered layout, nodes = {scale=0.75, align=center} ] {
      "a1\\ (1,2)" -> "q\\ (0,0)";
      "a2\\ (2,1)" -> "a1\\ (1,2)";
      "a3\\ (0,2)" -> "a2\\ (2,1)";
      "a4\\ (1,1)" -> "q\\ (0,0)";
      "a5\\ (2,1)" -> "a1\\ (1,2)";
      "a6\\ (0,1)" -> "a2\\ (2,1)";
      "a7\\ (1,2)" -> "q\\ (0,0)";
      "a8\\ (2,1)" -> "a1\\ (1,2)";
      "a9\\ (2,0)" -> "a6\\ (0,1)";
      };
      \end{tikzpicture}
      \caption{graphe juste avant la boucle}
      \label{fig:avant_boucle}
      \end{figure}

      \paragraph{}J'ai exécuté plusieurs fois le calcul du jeu sur ce graphe et dans certains cas assez rare le jeu s'arrête. Donc la convergence dépend de l'ordre dans lequel les joueurs votes.
\color{blue}
  \section{Nouveaux résultats sur les arbres}
    Suite à une analyse un peu plus appronfondie des exemples qui ne convergeaient pas la semaine dernière j'ai remarqué que cela était dû à une erreur dans mon code...

    Après une double correction voici quelques exemples d'arbres qui sont intéressant:
     


\color{black}
  \section{Questions}
    Je me pose une question à propos de la définition du prix de l'anarchie.

    On avait dit qu'on allait comparer le résultat du jeu avec la valeur trouvée si chacun des joueurs votent de la même façon que lors de la phase d'initialisation.

    Or j'ai remarqué que ce résultat peut être plus mauvais que l'anarchie et qu'en revanche une configuration totalement différente peut être meilleure. Il faudrait donc logiquement que je me place par rapport à cette meilleure solution pour chercher le prix de l'anarchie? non?\\

  \section{Ce qu'il faut que je fasse}

    \begin{itemize}
      \item Regarder le prix de l'anarchie.
      \item Continuer la recherche de convergence dans le cadre de la règle avec changement d'avis.
    \end{itemize}

\end{document}
