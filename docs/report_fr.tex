\documentclass[12pt]{article}

\usepackage{amsmath}
\usepackage{amssymb}
\usepackage[]{geometry}
\usepackage[francais]{babel}
\usepackage[utf8x]{luainputenc}
\usepackage{mathtools}
\usepackage{amsthm}
\usepackage{color}
\usepackage{graphicx}
\usepackage{tikz}
\usetikzlibrary{graphdrawing,graphs}
\usegdlibrary{layered}

\title{Compte rendu de la semaine}
\author{Alexis Martin}

\newtheoremstyle{not}
{\topsep}
{\topsep}
{}
{}
{\bfseries}
{.}
{\newline}
{}

\newtheoremstyle{defi}
{\topsep}
{\topsep}
{\itshape}
{}
{\bfseries}
{.}
{\newline}
{}

\newtheoremstyle{prob}
{\topsep}
{\topsep}
{}
{}
{\bfseries}
{.}
{\newline}
{}


\newtheorem{theoreme}{Theoreme}[section]
\newtheorem{lemme}{Lemme}[section]
\newtheorem{corollary}{Corrolaire}[section]
\newtheorem{proposition}{Proposition}[section]
\theoremstyle{defi}
\newtheorem{definition}{Définition}[section]
\theoremstyle{not}
\newtheorem{notation}{Notation}[section]
\theoremstyle{prob}
\newtheorem{problem}{Problème}[section]
\newtheorem{solution}{Solution}[problem]



\begin{document}
  \maketitle

  \section{Légende}
    \color{black}
      Ce qui est ancien et qui reste.

    \color{blue}
      Ce qui est a été ajouté cette semaine.

    \color{red}
      Ce qui est amené à disparaitre dans les semaines suivantes.

      P.S.: Le fichier Tex est sur Git, on aura donc toujours la possibilité de retrouver ce qui a été éffacé.

  \color{black}
  \section{Rappels}
    \subsection{Définitions}
      \begin{definition}[Graphe d'Argumentation]

        $F = \langle \mathcal{A}, \mathcal{R}, V \rangle$ avec :
        \begin{itemize}
          \item $\mathcal{A}$ : L'ensemble des arguments.
          \item $\mathcal{R} \subseteq \mathcal{A} \times \mathcal{A}$ : Une relation binaire entre arguments tels que : $(a,b) \in \mathcal{R}$ si $a$ "attaque" $b$.
          \item $V: \mathcal{A} \rightarrow \mathbb{N} \times \mathbb{N} :$ La fonction qui associe, à chaque argument, le nombre de "like" et de "dislike", $V(a) = (v^+, v^-)$ signifie que $a$ à $v^+$ "like" et $v^-$ "dislike".
        \end{itemize}
      \end{definition}

      \begin{notation}
        On note $Att(a)$ l'ensemble des arguments directs qui attaquent $a$ :
        $Att(a) = \{b\ |\ (b, a)\in \mathcal{R}\}$
      \end{notation}

      \begin{definition}
        \label{ref:def_tau}
        Soit $\tau_1$ la fonction défini par :
        $$
          \begin{array}{rclc}
            \tau_1 :  & \mathbb{N} \times \mathbb{N} & \longrightarrow & [0, 1] \\
            & (v^+,v^-) & \longmapsto & \frac{1}{2} \cdot \left(1 + sgn(v^+ - v^-) \cdot \left(1 - e^{\frac{-|v^+ - v^-|}{\varepsilon}}\right)\right)\\
          \end{array}
        $$

        où
        $sgn(x) = \left\{
          \begin{array}{ll}
            -1  & \mbox{si } x < 0 \\
            1 & \mbox{sinon} \\
          \end{array}
        \right.$
        et $\varepsilon = - \frac{\#joueurs}{ln(0.04)}$
      \end{definition}

      \paragraph{remarques/variantes}
      \label{ref:rem_def_tau}
        \begin{enumerate}
          \item On peut noter que cette fonction est la fonction de répartition de la première loi de Laplace.
          \item Avec cette définition on n'obtient pas une valeur de 1 (resp. $0$) sur l'argument $a$ si tous les agents "like" (resp. "dislikes"). On peut donc raffiner cette fonction et forcer ces cas limites :
            $$
              \begin{array}{rclc}
                \tau_2 :  & \mathbb{N} \times \mathbb{N} & \longrightarrow & [0, 1] \\
                & (\#joueurs, 0) & \longmapsto & 1 \\
                & (0, \#joueurs) & \longmapsto & 0 \\
                & (v^+,v^-) & \longmapsto & \frac{1}{2} \cdot \left(1 + sgn(v^+ - v^-) \cdot \left(1 - e^{\frac{-|v^+ - v^-|}{\varepsilon}}\right)\right)\\
              \end{array}
            $$\\
        \end{enumerate}

      \begin{proposition}
        Les fonctions $\tau$ sont croissantes selon $v^+$ et décroissantes selon $v^-$.
      \end{proposition}
      \begin{proof}
        On va montrer la croissance (resp. décroissance) selon $v^+$ (resp. $v^-$) pour la fonction $\tau_1$ et cette preuve sera suffisante pour se convaincre pour $tau_2$.

        Commençons par remarquer que :
        $$tau_1(v^+, v^-) = \left\{
          \begin{array}{ll}
            \frac{1}{2} \cdot \exp(\frac{v^+ - v^-}{\varepsilon})  & \mbox{si } v^+ - v^- < 0 \\
            1 - \frac{1}{2} \cdot \exp(\frac{v^- - v^+}{\varepsilon}) & \mbox{sinon} \\
          \end{array}
        \right.$$

        Soit $v^- \in \mathcal{N}$ et $v_1^+, v_2^+ \in \mathcal{N}$ tel que $v_1^+ < v_2^+$. Montrons que $tau_1(v_1^+, v^-) < tau_1(v_2^+, v^-)$

        3 cas possibles :
        \begin{itemize}
          \item si $v_1^+ - v^- < 0$ et $v_2^+ - v^- < 0$
          $$tau_1(v_1^+, v^-) - tau_1(v_2^+, v^-) = \frac{1}{2} \cdot \exp(\frac{- v^-}{\varepsilon})(\exp(\frac{v_1^+}{\varepsilon}) - \exp(\frac{v_2^+}{\varepsilon}))$$

          $\exp$ est une fonction croissante donc $tau_1(v_1^+, v^-) < tau_1(v_2^+, v^-)$

          \item si $v_1^+ - v^- < 0$ et $v_2^+ - v^- \geq 0$
          $$tau_1(v_1^+, v^-) - tau_1(v_2^+, v^-) = \frac{1}{2} \cdot \exp(\frac{v_1^+ - v^-}{\varepsilon}) - 1 + \frac{1}{2} \cdot \exp(\frac{v^- - v_2^+}{\varepsilon})$$

          En remarquant que $\exp(\frac{v_1^+ - v^-}{\varepsilon}) < 1$ et que $\exp(\frac{v^- - v_2^+}{\varepsilon}) \leq 1$ on en conclut que $tau_1(v_1^+, v^-) < tau_1(v_2^+, v^-)$

          \item si $v_1^+ - v^- \geq 0$ et $v_2^+ - v^- \geq 0$
          $$tau_1(v_1^+, v^-) - tau_1(v_2^+, v^-) = \frac{1}{2} \cdot \exp(\frac{v^-}{\varepsilon})(\exp(\frac{- v_2^+}{\varepsilon}) - \exp(\frac{- v_1^+}{\varepsilon}))$$

          $\exp(\frac{- v_2^+}{\varepsilon}) < \exp(\frac{- v_1^+}{\varepsilon})$ donc $tau_1(v_1^+, v^-) < tau_1(v_2^+, v^-)$
        \end{itemize}

        La preuve est la même pour la décroissance selon $v^-$.
      \end{proof}
\color{black}
      On peut ensuite définir une sémantique d'argumentation.
      \begin{definition}[Laplace Product Semantics]
        $\mathcal{S}_1$ (resp $\mathcal{S}_2$) $= \langle [0, 1], \tau_\varepsilon, \curlywedge, \curlyvee, \neg  \rangle$ est la sémantique tel que :
        \begin{enumerate}
          \item $x_1 \curlywedge x_2 = x_1 \cdot x_2$
          \item $x_1 \curlyvee x_2 = x_1 + x_2 - x_1 \cdot x_2$
          \item $\neg x_1 = 1 - x_1$
          \item $\tau_1$ (resp. $\tau_2$) la fonction défini précédement.
        \end{enumerate}
      \end{definition}

      On notera par la suite $\mathcal{S}$ lorsque $\mathcal{S}_1$ ou $\mathcal{S}_2$ peuvent être utilisée indifférement.
      De même on notera $\tau$ pour désigner $\tau_1$ ou $\tau_2$.

      \begin{definition}[modèle social]
        Soit $F= \langle \mathcal{A}, \mathcal{R}, V \rangle$ un graphe d'argumentation et $\mathcal{S} = \langle [0, 1], \tau, \curlywedge, \curlyvee, \neg  \rangle$.
        La fonction $LM : \mathcal{A} \rightarrow [0, 1]$ est un $\mathcal{S}$-Model pour $F$ si :

        $LM(a) = \tau(a) \curlywedge \neg$ {\Large $\curlyvee$} $\{LM(a_i)\ |\ a_i \in Att(a)\}$
      \end{definition}

  \section{Les modèles}
    Soit $F = \langle \mathcal{A}, \mathcal{R}, V \rangle$ un graphe d'argumentation, $\mathcal{J} = \{1, \ldots, n\}$ $n$ agents ($=$ joueurs) et $\mathcal{S}$ la sémantique utilisée.
    Chaque agent connait tout le graphe $F$, ils connaissent donc tous les arguments et la question.

    \paragraph{Initialisation}
      \begin{enumerate}
        \item Chaque agent $k$ analyse le débat, il "like" les arguments qu'il trouve recevables et "dislike" ceux qu'il juge inconsistant.
        Les arguments recevables auront donc 1 "like" et 0 "dislike", ceux inconsistant auront 0 "like" et 1 "dislike" et les autres 0 "like" et "dislike".
        \item En tenant compte de ces poids il calcule la valeur qu'il associe à la question. Cette valeur n'est pas connu des autres agents.
      \end{enumerate}
      \begin{definition}
        Soit $F_k = \langle \mathcal{A}, \mathcal{R}, V_k \rangle$ le graphe d'argumentation du joueur $k$.
        La fonction $V_k$ définie donc la préférence du joueur $k$ sur chacun des arguments de $\mathcal{A}$.
      \end{definition}
      \begin{definition}
        Soit $LM_k$ un $\mathcal{S}$-Model pour le graphe $F_k$.

        On appelle $LM_\mathcal{J}$ la fonction qui associe à chaque joueur la valeur qu'il associe à la question.

        $$
          \begin{array}{rclc}
            LM_\mathcal{J} :  & \mathcal{J} & \longrightarrow & [0, 1] \\
            & k & \longmapsto & LM_k(q)\\
          \end{array}
        $$
      \end{definition}


    \paragraph{Le jeu}(deux variantes possibles)
      \begin{description}
        \item[Variante 1] Les agents se regroupent autour d'une table où seul la question est exposée. Ils reconstruisent le débat en ajoutant les arguments les uns après les autres. Pour ce faire lorsqu'un agent vote ("like" ou "dislike") sur un argument non présent, l'argument s'ajoute à la représentation avec le vote.

        \item[Variante 2] Les agents se regroupent autour d'une table où une représentation du débat s'y trouve. Cette représentation est dépourvu de poids ("like" ou "dislike") et la valeur de la question dans cette représentation est affiché et connu de tous.
      \end{description}

    \paragraph{objectif}(deux objectifs différents possibles)
      \begin{description}
        \item[Objectif 1] L'objectif pour chacun des joueurs va être de faire tendre la valeur de la question de cette représentation vers sa valeur calculée lors de la phase d'initialisation.
        \item[Objectif 2] L'intervalle $[0, 1]$ est divisé en cluster. Un joueur appartient à un cluster si sa valeur appartient au cluster.

        L'objectif de chaque joueur est que la valeur finale du débat appartienne à son cluster. Les joueurs peuvent alors former des alliances.
      \end{description}


    \paragraph{Règles}(deux variantes possibles)
      \begin{description}
        \item[Variante 1] Chaque agent peut voter qu'une fois (k fois) par argument.

        \item[Variante 2] Les agents peuvent changer d'avis sur le vote qu'ils ont attribué à un argument et donc retirer leur vote ou l'inverser.
      \end{description}

    \paragraph{Dynamique\\}
      On considère dans ces modèles une dynamique de meilleure réponse. Round Robin.

    \paragraph{Les questions qu'on se pose}
      \begin{itemize}
        \item Il y a t-il convergence?
        \item Si tous les agents ont la même valeur à la fin de l'initialisation, il y a t-il convergence? Il existe alors un équilibre mais peut-on toujours l'atteindre? (Ce n'est pas la même question? Si on peut toujours atteindre l'équilibre alors ça converge, et si ça converge alors il y a un équilibre? Non?)
        \item Quel est le prix de l'anarchie dans le cas de la règle 1?
        \item Peut-on trouver une stratégie pour les agents?
      \end{itemize}



  \section{Problèmes liés aux modèles}
    \begin{problem}
      Dans notre cas, on considère que la question (notons la $q$) ne reçoit aucun vote (ce qui est plutôt intuitif). Mais si on prend cette définition pour la sémantique alors $LM(q) = 0$ quelques soit les attaquants car $\tau_\varepsilon(q) = 0$.
    \end{problem}

    \begin{solution}
      On a changé la définition de $\tau$ cf. définition \ref{ref:def_tau} page \pageref{ref:def_tau}
    \end{solution}

    \begin{problem}
      Un autre problème un peu similaire au précédent.
      D'après la définition de Leite et Martins un argument qui n'a pas de vote a un poids à 0 (i.e. $\tau_\varepsilon (a) = 0$) et un vote qui n'a que des votes négatif a également un poids à 0.

      Le problème dans l'initialisation (entre autre) est que "disliker" un argument (ne pas le trouver recevable du tout) ou ne pas avoir d'avis reviens au même, ce fait ce propage à tous les fils de ces arguments.

      Cela peut paraitre incohérent.
    \end{problem}

    \begin{solution}
      La nouvelle définition de $\tau$ (cf. définition \ref{ref:def_tau} page \pageref{ref:def_tau}) corrige également cette incohérence.
    \end{solution}


  \section{Tests et Exemples}
    \subsection{Le cas des Arbres}
      \subsubsection{Question: Peut-on trouver des arbres ou les agents n'ont pas les mêmes préférences mais la même valeur à la fin de l'initialisation?\newline}

        \paragraph{Si on a la fonction $\tau_2$}

          \subparagraph{rappel} Cette fonction $\tau_2$ force les cas limites (c.f. remarque de la definition \ref{ref:def_tau}) \\

          Sur le débat représenté dans la figure \ref{fig:dif_pref} on remarque que le joueur 1 et le joueur 2 ont des avis totalement différents sur chacun des arguments. En revanche la valeur de la question est la même. A savoir : $LM_1(q) = LM_2(q) = 0.125$

          \begin{figure}
            \centering
            \begin{tabular}{ccc}
              \begin{tikzpicture}[>=stealth]
              \graph [ layered layout, nodes = {scale=0.75, align=center} ] {
              "a1\\ (0,0)" -> "q\\ (0,0)";
              "a2\\ (0,0)" -> "q\\ (0,0)";
              "a3\\ (0,0)" -> "a1\\ (0,0)";
              "a4\\ (0,0)" -> "a1\\ (0,0)";
              "a5\\ (0,0)" -> "a2\\ (0,0)";
              "a6\\ (0,0)" -> "a2\\ (0,0)";
              "a7\\ (0,0)" -> "a3\\ (0,0)";
              "a8\\ (0,0)" -> "a3\\ (0,0)";
              };
              \end{tikzpicture} &
              \begin{tikzpicture}[>=stealth]
              \graph [ layered layout, nodes = {scale=0.75, align=center} ] {
              "a1\\ (1,0)" -> "q\\ (0,0)";
              "a2\\ (0,1)" -> "q\\ (0,0)";
              "a3\\ (1,0)" -> "a1\\ (1,0)";
              "a4\\ (0,1)" -> "a1\\ (1,0)";
              "a5\\ (1,0)" -> "a2\\ (0,1)";
              "a6\\ (1,0)" -> "a2\\ (0,1)";
              "a7\\ (0,0)" -> "a3\\ (1,0)";
              "a8\\ (0,0)" -> "a3\\ (1,0)";
              };
              \end{tikzpicture} &
              \begin{tikzpicture}[>=stealth]
              \graph [ layered layout, nodes = {scale=0.75, align=center} ] {
              "a1\\ (0,0)" -> "q\\ (0,0)";
              "a2\\ (0,0)" -> "q\\ (0,0)";
              "a3\\ (0,1)" -> "a1\\ (0,0)";
              "a4\\ (0,1)" -> "a1\\ (0,0)";
              "a5\\ (0,1)" -> "a2\\ (0,0)";
              "a6\\ (0,1)" -> "a2\\ (0,0)";
              "a7\\ (1,0)" -> "a3\\ (0,1)";
              "a8\\ (1,0)" -> "a3\\ (0,1)";
              };
              \end{tikzpicture} \\
            \end{tabular}

            \caption{De gauche à droite: graphe général, préférences du joueur 1, préférences du joueur 2}
            \label{fig:dif_pref}
          \end{figure}

        \paragraph{Si l'on considère la fonction $\tau_1$\\}
          Je n'ai pas encore trouvé d'exemple qui possède cette propriété.
          Cela est dû à la définition de $\tau_1$, un argument "disliker" n'aura pas un poids égale à 0 et un argument "liker" n'aura pas un poids égale à 1.

      %TODO Modifier cette partie sur l'équilibre. En discuter avec Nicolas et Elise
      \subsubsection{Règle du jeu : ONESHOT Question: Si tous les agents ont la même valeur à la fin de l'initialisation, peut-on atteindre la valeur des agents ?}

        \paragraph{Si l'on considère la fonction $\tau_2$\\}

          Sur l'exemple précédent (c.f. figure \ref{fig:dif_pref}) on atteint la valeur des agents.

          Sur l'exemple trivial de la figure \ref{fig:trivial} on atteint également la valeur des agents.

          \begin{figure}
            \centering
            \begin{tabular}{cccc}
              \begin{tikzpicture}[>=stealth]
              \graph [ layered layout, nodes = {scale=0.75, align=center} ] {
              "a1\\ (0,0)" -> "q\\ (0,0)";
              "a2\\ (0,0)" -> "q\\ (0,0)";
              "a3\\ (0,0)" -> "q\\ (0,0)";
              };
              \end{tikzpicture} &

              \begin{tikzpicture}[>=stealth]
              \graph [ layered layout, nodes = {scale=0.75, align=center} ] {
              "a1\\ (1,0)" -> "q\\ (0,0)";
              "a2\\ (0,1)" -> "q\\ (0,0)";
              "a3\\ (0,1)" -> "q\\ (0,0)";
              };
              \end{tikzpicture} &

              \begin{tikzpicture}[>=stealth]
              \graph [ layered layout, nodes = {scale=0.75, align=center} ] {
              "a1\\ (0,1)" -> "q\\ (0,0)";
              "a2\\ (1,0)" -> "q\\ (0,0)";
              "a3\\ (0,1)" -> "q\\ (0,0)";
              };
              \end{tikzpicture} &

              \begin{tikzpicture}[>=stealth]
              \graph [ layered layout, nodes = {scale=0.75, align=center} ] {
              "a1\\ (0,1)" -> "q\\ (0,0)";
              "a2\\ (0,1)" -> "q\\ (0,0)";
              "a3\\ (1,0)" -> "q\\ (0,0)";
              };
              \end{tikzpicture} \\
            \end{tabular}

            \caption{De gauche à droite: graphe général, préférences du joueur 1, préférences du joueur 2, préférences du joueur 3}
            \label{fig:trivial}
          \end{figure}

          \subparagraph{Remarque\\}
            Sur cet exemple, il est intéressant de noter que $LM_1(q) = LM_2(q) = LM_3(q) = 0$.
            En effet $LM(q) = 0.5 \cdot (1 - \curlyvee \{LM(a) | a\in Att(q)\})$ et $\curlyvee \{LM(a) | a\in Att(q)\} = 1$


        \paragraph{Si l'on considère $\tau_1$\\}
          A priori, on n'atteint pas la valeur des agents.

          Sur l'exemple de la figure \ref{fig:dif_pref} on atteint la valeur des agents, en revanche sur la figure \ref{fig:trivial}, on ne l'atteint pas ($LM_1(q) = LM_2(q) = LM_3(q) = 0.009604$ alors que $LM_g(q) = 0.00731$).
\color{red}
          \subparagraph{remarque\\} Je me demande si cela n'est pas dû à une précision machine car le système se stabilise dans une position ou un coup est très proche d'être meilleur.
            Je ferais le détail des calculs la semaine prochaine pour être sur.
\color{black}
      \subsubsection{Question: Notre définition de $\tau$ change t'elle radicalement les résultats obtenus sur les exemples de Leite et Martins?}

        \paragraph{Le premier exemple\\}

          \begin{figure}
            \centering
            \begin{tikzpicture}[>=stealth]
              \graph [ layered layout, nodes = {scale=0.75, align=center} ] {
              "a\\ (20,20)" -> "b\\ (20,20)";
              "b\\ (20,20)" -> "a\\ (20,20)";
              "c\\ (60,10)" -> "b\\ (20,20)";
              "e\\ (40,10)" -> "c\\ (60,10)";
              "d\\ (10,40)" -> "c\\ (60,10)";
              "e\\ (40,10)" -> "d\\ (10,40)";
              "d\\ (10,40)" -> "e\\ (40,10)";
              };
            \end{tikzpicture}
            \caption{Exemple de l'article Social Abstract Argumentation de Leite et Martins}
            \label{fig:LM_example}
          \end{figure}

          L'exemple de Leite et Martins est représenté figure \ref{fig:LM_example} et le tableau ci-dessous référence les valeurs de chacun des arguments trouvés par Leite et Martins ainsi que les notres.

          \begin{tabular}{|c|c|c|c|c|c|}
            \hline
                       & a    & b    & c    & d     & e \\
            \hline
            LM         & 0.37 & 0.25 & 0.19 & 0.05  & 0.76 \\
            \hline
            def $\tau$ & 0.36 & 0.28 & 0.13 & 0.018 & 0.86\\
            \hline
          \end{tabular}

          Il n'y a donc pas de différence fondamentale, les arguments sont toujours classés dans le même ordre.

        \paragraph{Le deuxième exemple\\}
          Cet exemple est tiré de l'article Extending Social Abstract Argumentation with Votes on Attacks.
          Il est donné figure \ref{fig:LM2_example}. Les résultats observés sont :

          \begin{figure}
            \centering
            \begin{tikzpicture}[>=stealth]
              \graph [ layered layout, nodes = {scale=0.75, align=center} ] {
              "a\\ (70,70)" -> "b\\ (70,70)";
              "b\\ (70,70)" -> "a\\ (70,70)";
              "c\\ (54,66)" -> "a\\ (70,70)";
              "e\\ (19,1)" -> "a\\ (70,70)";
              "d\\ (130,61)" -> "c\\ (54,66)";
              "d\\ (130,61)" -> "b\\ (70,70)";
              };
            \end{tikzpicture}
            \caption{Deuxième exemple de Leite et Martins}
            \label{fig:LM2_example}
          \end{figure}

          \begin{tabular}{|c|c|c|c|c|c|}
            \hline
                       & a    & b     & c     & d    & e \\
            \hline
            LM         & 0.02 & 0.16  & 0.14  & 0.68 & 0.95 \\
            \hline
            def $\tau$ & 0.16 & 0.066 & 0.063 & 0.84 & 0.63\\
            \hline
          \end{tabular}

          En revanche sur cet exemple les changements sont plus importants.

          Avec la défintion de Leite et Martins on a : $a \prec c \prec b \prec d \prec e$ alors qu'avec notre définition on a : $c \prec b \prec a \prec e \prec d$


  \section{Résultats}
  \color{red}
    \subsection{Atteint-on l'équilibre avec la première définition de $\tau$ dans l'exemple de la figure \ref{fig:trivial}?}

      \paragraph{remarque :}
        Je viens de me rendre compte que j'ai peut être mal interprété le terme équilibre.
        J'entend par équilibre le fait d'atteindre la valeur exact souhaité par les 3 joueurs (dans le cadre de l'exemple de la figure \ref{fig:trivial}).
        C'est à dire 0.009604.

        La réponse est non, j'ai refait le calcul exact à la main et on atteint pas 0.009604 avec ce point de départ.\\

    En revanche cet objectif existe, si on prend :
    \begin{itemize}
      \item a1 : $likes = 3, dislikes = 0$
      \item a2 : $likes = 0, dislikes = 3$
      \item a3 : $likes = 0, dislikes = 3$
    \end{itemize}

    Alors $LM(q) = 0.009604$\\

    En changeant le nombre de vote au départ, par exemple si on prend :
    \begin{itemize}
      \item a1 : $likes = 1, dislikes = 2$
      \item a2 : $likes = 1, dislikes = 2$
      \item a3 : $likes = 1, dislikes = 2$
    \end{itemize}

    on atteint pas non plus l'équilibre, on reste bloqué comme avant à environ $0.00731$.
  \color{black}
  \section{Nouveaux résultats sur les arbres}
  \color{red}
    Suite à une analyse un peu plus appronfondie des exemples qui ne convergeaient pas la semaine dernière j'ai remarqué que cela était dû à une erreur dans mon code...
  \color{black}
    Après une double correction voici quelques exemples d'arbres qui sont intéressant:\\

    \begin{itemize}
      \item Le premier exemple est un jeu assez général comportant 10 noeuds et 3 joueurs.
        Ce jeu est représenté figure \ref{fig:3_players_10_vertices}

        Lorsque l'on considère la fonction $\tau_1$, chaque joueur à une préférence différente pour la question.
        \begin{description}
          \item[Joueur 1 : ] $LM_1(q) = 0.46622$
          \item[Joueur 2 : ] $LM_2(q) = 0.73485$
          \item[Joueur 3 : ] $LM_3(q) = 0.03879$
        \end{description}

        Sur le graphique de la fonction $\tau_1$ on remarque que la valeur converge vers la valeur médiane des joueurs donc vers le joueur du milieu. Cette propriété semble se vérifier lorsque la valeur commence entre $j_min$ et $j_max$.

        Dans ce jeu il a également eu 7 coups ou un joueur a décidé de ne pas jouer et il y a eu 1 changement d'avis.

      \item Le deuxième est un petit exemple montrant un cas ou la valeur n'est pas comprise entre les valeurs min et max des joueurs.
        Il est représenté sur la figure \ref{fig:not_in_range}.

        On remarque que le joueur 1 a voté de manière a avoir la valeur du graphe la plus forte possible.

        En revanche pour le joueur 2 des coups peuvent encore permettre d'augmenter la valeur du graphe mais ces coups ne sont pas meilleurs que la valeur actuelle pour le joueur 2.
    \end{itemize}

    \begin{figure}
      \centering
      \begin{tabular}{cc}
        \begin{tikzpicture}[>=stealth]
        \graph [ layered layout, nodes = {scale=0.75, align=center} ] {
        "a1\\ (0,0)" -> "q\\ (0,0)";
        "a2\\ (0,0)" -> "q\\ (0,0)";
        "a3\\ (0,0)" -> "q\\ (0,0)";
        "a4\\ (0,0)" -> "a2\\ (0,0)";
        "a5\\ (0,0)" -> "a1\\ (0,0)";
        "a6\\ (0,0)" -> "q\\ (0,0)";
        "a7\\ (0,0)" -> "a6\\ (0,0)";
        "a8\\ (0,0)" -> "a6\\ (0,0)";
        "a9\\ (0,0)" -> "a2\\ (0,0)";
        "a10\\ (0,0)" -> "a6\\ (0,0)";
        };
        \end{tikzpicture} &

        \begin{tikzpicture}[>=stealth]
        \graph [ layered layout, nodes = {scale=0.75, align=center} ] {
        "a1\\ (1,0)" -> "q\\ (0,0)";
        "a2\\ (0,0)" -> "q\\ (0,0)";
        "a3\\ (0,1)" -> "q\\ (0,0)";
        "a4\\ (1,0)" -> "a2\\ (0,0)";
        "a5\\ (0,0)" -> "a1\\ (1,0)";
        "a6\\ (0,0)" -> "q\\ (0,0)";
        "a7\\ (0,0)" -> "a6\\ (0,0)";
        "a8\\ (0,0)" -> "a6\\ (0,0)";
        "a9\\ (0,0)" -> "a2\\ (0,0)";
        "a10\\ (0,0)" -> "a6\\ (0,0)";
        };
        \end{tikzpicture} \\

        \begin{tikzpicture}[>=stealth]
        \graph [ layered layout, nodes = {scale=0.75, align=center} ] {
        "a1\\ (0,0)" -> "q\\ (0,0)";
        "a2\\ (0,1)" -> "q\\ (0,0)";
        "a3\\ (0,1)" -> "q\\ (0,0)";
        "a4\\ (1,0)" -> "a2\\ (0,1)";
        "a5\\ (0,0)" -> "a1\\ (0,0)";
        "a6\\ (0,1)" -> "q\\ (0,0)";
        "a7\\ (0,1)" -> "a6\\ (0,1)";
        "a8\\ (1,0)" -> "a6\\ (0,1)";
        "a9\\ (1,0)" -> "a2\\ (0,1)";
        "a10\\ (0,0)" -> "a6\\ (0,1)";
        };
        \end{tikzpicture} &

        \begin{tikzpicture}[>=stealth]
        \graph [ layered layout, nodes = {scale=0.75, align=center} ] {
        "a1\\ (1,0)" -> "q\\ (0,0)";
        "a2\\ (0,1)" -> "q\\ (0,0)";
        "a3\\ (0,1)" -> "q\\ (0,0)";
        "a4\\ (0,1)" -> "a2\\ (0,1)";
        "a5\\ (0,1)" -> "a1\\ (1,0)";
        "a6\\ (0,1)" -> "q\\ (0,0)";
        "a7\\ (1,0)" -> "a6\\ (0,1)";
        "a8\\ (0,1)" -> "a6\\ (0,1)";
        "a9\\ (1,0)" -> "a2\\ (0,1)";
        "a10\\ (0,0)" -> "a6\\ (0,1)";
        };
        \end{tikzpicture} \\

        \begin{tikzpicture}[>=stealth]
        \graph [ layered layout, nodes = {scale=0.75, align=center} ] {
        "a1\\ (0,0)\\ (0,1)\\ (1,0)" -> "q\\ (0,0)\\lm = 0.46209";
        "a2\\ (0,0)\\ (0,1)\\ (1,0)" -> "q\\ (0,0)\\lm = 0.46209";
        "a3\\ (0,1)\\ (0,1)\\ (1,0)" -> "q\\ (0,0)\\lm = 0.46209";
        "a4\\ (0,0)\\ (1,0)\\ (0,1)" -> "a2\\ (0,0)\\ (0,1)\\ (1,0)";
        "a5\\ (0,0)\\ (1,0)\\ (0,1)" -> "a1\\ (0,0)\\ (0,1)\\ (1,0)";
        "a6\\ (0,0)\\ (0,1)\\ (1,0)" -> "q\\ (0,0)\\lm = 0.46209";
        "a7\\ (0,0)\\ (1,0)\\ (0,1)" -> "a6\\ (0,0)\\ (0,1)\\ (1,0)";
        "a8\\ (0,0)\\ (1,0)\\ (0,1)" -> "a6\\ (0,0)\\ (0,1)\\ (1,0)";
        "a9\\ (0,1)\\ (1,0)\\ (0,1)" -> "a2\\ (0,0)\\ (0,1)\\ (1,0)";
        "a10\\ (0,0)\\ (1,0)\\ (0,1)" -> "a6\\ (0,0)\\ (0,1)\\ (1,0)";
        };
        \end{tikzpicture} &

        \begin{tikzpicture}[>=stealth]
        \graph [ layered layout, nodes = {scale=0.75, align=center} ] {
        "a1\\ (1,0)\\ (0,1)\\ (1,0)" -> "q\\ (0,0)\\lm = 0.11241";
        "a2\\ (1,0)\\ (0,1)\\ (1,0)" -> "q\\ (0,0)\\lm = 0.11241";
        "a3\\ (1,0)\\ (0,1)\\ (1,0)" -> "q\\ (0,0)\\lm = 0.11241";
        "a4\\ (0,1)\\ (1,0)\\ (0,1)" -> "a2\\ (1,0)\\ (0,1)\\ (1,0)";
        "a5\\ (0,1)\\ (1,0)\\ (0,1)" -> "a1\\ (1,0)\\ (0,1)\\ (1,0)";
        "a6\\ (1,0)\\ (0,1)\\ (1,0)" -> "q\\ (0,0)\\lm = 0.11241";
        "a7\\ (0,1)\\ (1,0)\\ (0,1)" -> "a6\\ (1,0)\\ (0,1)\\ (1,0)";
        "a8\\ (0,1)\\ (1,0)\\ (0,1)" -> "a6\\ (1,0)\\ (0,1)\\ (1,0)";
        "a9\\ (0,1)\\ (1,0)\\ (0,1)" -> "a2\\ (1,0)\\ (0,1)\\ (1,0)";
        "a10\\ (0,1)\\ (1,0)\\ (0,1)" -> "a6\\ (1,0)\\ (0,1)\\ (1,0)";
        };
        \end{tikzpicture} \\

        \includegraphics[scale=0.35]{/home/talkie/Documents/Stage/DebateGames/docs/examples/game_10_tau1.png}&
        \includegraphics[scale=0.35]{/home/talkie/Documents/Stage/DebateGames/docs/examples/game_10_LM.png}
      \end{tabular}

      \caption{De gauche à droite: graphe initial, préférences du joueur 1, préférences du joueur 2, préférences du joueur 3, graphe avec la fonction $\tau_1$, graphe avec la fonction $LM$, courbe avec la fonction $\tau_1$, courbe avec la fonction $LM$}
      \label{fig:3_players_10_vertices}
    \end{figure}

    \begin{figure}
      \centering
      \begin{tabular}{cc}
        \begin{tikzpicture}[>=stealth]
        \graph [ layered layout, nodes = {scale=0.75, align=center} ] {
        "a1\\ (0,1)" -> "q\\ (0,0)\\lm = 0.99980008";
        "a2\\ (1,0)" -> "a1\\ (0,1)";
        "a3\\ (1,0)" -> "a1\\ (0,1)";
        "a4\\ (0,1)" -> "a1\\ (0,1)";
        "a5\\ (0,0)" -> "a3\\ (1,0)";
        };
        \end{tikzpicture} &

        \begin{tikzpicture}[>=stealth]
        \graph [ layered layout, nodes = {scale=0.75, align=center} ] {
        "a1\\ (1,0)" -> "q\\ (0,0)\\lm = 0.99039208";
        "a2\\ (1,0)" -> "a1\\ (1,0)";
        "a3\\ (0,1)" -> "a1\\ (1,0)";
        "a4\\ (0,0)" -> "a1\\ (1,0)";
        "a5\\ (0,1)" -> "a3\\ (0,1)";
        };
        \end{tikzpicture} \\

        \begin{tikzpicture}[>=stealth]
        \graph [ layered layout, nodes = {scale=0.75, align=center} ] {
        "a1\\ (0,1)\\ (1,0)" -> "q\\ (0,0)\\lm = 0.98725";
        "a2\\ (1,0)\\ (0,1)" -> "a1\\ (0,1)\\ (1,0)";
        "a3\\ (1,0)\\ (1,0)" -> "a1\\ (0,1)\\ (1,0)";
        "a4\\ (1,0)\\ (0,0)" -> "a1\\ (0,1)\\ (1,0)";
        "a5\\ (0,1)\\ (1,0)" -> "a3\\ (1,0)\\ (1,0)";
        };
        \end{tikzpicture} &

        \includegraphics[scale=0.35]{/home/talkie/Documents/Stage/DebateGames/docs/examples/not_in_range_tau_1.png}
      \end{tabular}

      \caption{De gauche à droite: préférences du joueur 1, préférences du joueur 2, graphe avec la fonction $\tau_1$, courbe avec la fonction $\tau_1$}
      \label{fig:not_in_range}
    \end{figure}
    \color{red}
    \paragraph{remarque/intuition}
      Pour montrer la convergence, c'est suffisant de traiter chacun des joueurs séparement dans l'évolution du système.
      J'entend par là qu'il est suffisant de montrer que pour chaque joueur (et quelque soit les votes des autres joueurs) on arrive à une situation d'équilibre pour ce joueur.
\color{blue}
    \subsection{quelques résultats observés sur les exemples}
      \paragraph{jeux à plus de deux joueurs}
      \begin{itemize}
        \item Si on a un nombre pair de joueurs et que la valeur commence au milieu des joueurs alors le jeu à l'air de s'équilibrer proche de la valeur initiale.
        \item si on a un nombre impair de participant et que la valeur commence au milieu des joueurs alors le jeu à l'air de converger vers la valeur médiane des joueurs.
      \end{itemize}

      Il est possible qu'il existe un argument similaire à celui de la preuve de convergence des jeux à deux joueurs exposé ci-dessous.
      Le problème est que le nombre de cas à traiter est plus conséquent et que je n'ai pas encore trouver une généralisation de ces cas afin de tous les traiter en même temps.

      Dans les exemples, les joueurs extrêmes ont l'air de s'opposer leur votes et, dans le cas d'un nombre impair de joueurs, seul le joueur du milieu aurait un impact sur l'évolution de la valeur du jeu.
\color{blue}
  \section{Convergence des jeux à 2 joueurs sur les arbres}
    \subsection{Rappel des questions qu'on se pose}
      \begin{enumerate}
        \item A 1 joueur, la valeur finale du graphe est toujours égale à son avis personnel. FAUX
        \item Un joueur extrême ne peux jamais changer un de ses votes. VRAI
        \item Tout états d'équilibre est entre le min et le max. FAUX
        \item Il existe au moins 1 équilibre entre min et max. OUVERT
        \item A deux joueurs, si la valeur initiale est entre min et max alors on converge vers cette valeur. VRAI
        \item Soit $d_i$ la distance à la valeur initiale pour l'agent i. Tout équilibre doit être à une distance inférieure à $d_i$ pour tout agents. VRAI jusqu'à 2 joueurs
      \end{enumerate}

    \subsection{Notation}
      Commençons par donner quelques notations :

      On note $PRO$ l'ensemble des arguments qui défendent la question et $CON$ les arguments qui attaquent la question.

      Par extension, on note $PRO^v_j$ avec $v \in \{-, 0, +\}$ l'ensemble des arguments sur lesquels l'agent $j$ a votés pour ($v = +$), contre ($v = -$) ou c'est abstenu ($v = 0$).

      On note également $v_0$ la valeur initiale du débat, la valeur du jeu avant que le premier vote eut lieu, $v_f$ la valeur finale du débat, la valeur du jeu lorsqu'aucun joueur ne peut jouer et enfin $v_i$ la valeur du jeu à l'instant $i$, avec $i \in [\![ 0;f ]\!]$.

      Enfin, étant donné qu'on a que 2 joueurs on notera $j_{min}$ le joueur ayant la valeur minimum noté $v_{min}$ et $j_{max}$ celui qui à la plus grande noté $v_{max}$.

      \begin{definition}
        On appelle $\sigma_i \in (\mathbb{N} \times \mathbb{N})^{|\mathcal{A}|}$ la configuration à l'instant $i$ du jeu.
      \end{definition}

      $\sigma_i$ est donc un vecteur qui correspond aux votes (positifs et négatifs) sur tous les arguments à l'instant $i$.

      \paragraph{remarque} Sur les arbres : $PRO \cap CON = \emptyset$.

    \subsection{Quelques faits intéressants}
      \paragraph{A un joueur} on peut extraire une fonction potentielle pour le jeu :
        $$ \Phi(i) = |v_j - v_i| $$

        avec $v_j$ la valeur de l'unique joueur.

        On peut facilement montrer que à chaque réponse améliorante la fonction $\Phi$ décroit.

      \paragraph{Recherche de fonctions potentielles pour le jeu à 2 joueurs}

        L'objectif est de trouver une fonction qui va croitre ou décroitre à chaque coup améliorant.

        \begin{itemize}
          \item La somme des distances entre la valeur et les joueurs ne fonctionne pas.
            Cette fonction peut croitre ou décroitre selon le coup améliorant.

          \item il faut trouver une fonction qui croît (ou décroît) malgré le fait que $v_f$ peut être égale à $v_0$.

            Il faut donc inclure les votes des arguments ?

            Serait-il possible via une fonction de quantifier le nombre de configurations améliorantes pour l'un des deux joueurs? Une fonction qui nous dirait : depuis cette configuration $i$ je ne pourrait plus jamais atteindre la configuration $j$ et donc toutes les configurations qui en descendent.

            Si à chaque coup améliorant on fait descendre le nombre de situations, alors on a gagné.

          \item En better response le temps qu'il faut pour trouver un équilibre est plus long. Mais on tombe à priori toujours sur un équilibre.

          Il faudrait regarder si on cycle de temps en temps!

          \item En random les mêmes constatations qu'en better response sont observées.
        \end{itemize}
    \subsection{Preuve de la question 6}
    \begin{theoreme}
    \label{thm:question_6}
      Soit $d_i$ la distance entre l'agent $i$ et $v_0$. A deux joueurs, tout point d'équilibre est à une distance inférieure ou égale à $d_i$ pour tout agent $i$.
    \end{theoreme}

    \begin{proof}
      Supposons que l'état $s$ est stable et que $|v_s - v_0| > d_i$ pour l'agent $i$.
      Il y a deux cas : soit $v_s$ est supérieur à $v_0$ soit $v_s$ est inférieur, SPDG supposons qu'il est inférieur.

      On rappel que l'agent $i$ peut obtenir exactement $v_0$ en annulant tous les coups de son adversaire $j$.
      Comme $v_s<v_0$, on a :
      $$
      C = (PRO^-_j \setminus PRO^+_i) \cup (CON^+_j \setminus CON^-_i) \cup PRO^-_i \cup CON^+i \not = \emptyset
      $$
      (Intuitivement, cet ensemble contient tous les votes que l'agent $i$ peut faire pour obtenir $v_0$: soit voter pour un argument $PRO$ sur lequel $j$ à voter contre, ou voter contre un argument $CON$ sur lequel $j$ à voter pour, ou s'abstenir sur un argument sur lequel $j$ n'a pas réagi).

      Jouer un de ces coups augmentera la valeur du débat et jouer tous les coups ammenera la valeur du débat à la valeur initiale $v_0$. Donc tous les coups de $C$ sont des coups améliorants, donc $s$ n'est pas un état stable.
    \end{proof}

    \begin{corollary}
      $v_0$ est l'\emph{unique} equilibre lorsque $v_0 \in [v_{min},v_{max}]$
    \end{corollary}

    \begin{proof}
      Par contradiction, supposons qu'il existe un autre equilibre $v$.
      Soit $v$ est supérieur à $v_0$, soit il est inférieur.
      Supposons SPDG qu'il est inférieur.
      Donc la distance $d_{max}$ entre le joueur $j_{max}$ et $v_0$ est inférieur à la distance entre $j_{max}$ et $v$.
      D'après le théorème \ref{thm:question_6} c'est impossible, donc $v$ n'est pas un équilibre.
    \end{proof}

    \subsection{Preuve de la question 5}

      \begin{theoreme}
        \label{thm:question_5}
        A deux joueurs, tous les systèmes convergent avec la dynamique Round Robin.
      \end{theoreme}

      % \begin{proof}
      %   essayons de montrer qu'il n'y a pas de cycle. pour qu'un joueur change d'avis sur un argument il faut que la valeur switch entre supérieur et inférieur à sa valeur.
      %
      %   Prenons la liste des situations possibles.
      %   montrons qu'il ne peut pas y avoir plus de switch dans le jeu que $min(sup, inf)$.
      %
      %   supposons qu'il y a plus de switch. Dans ce cas on a vu 2 fois la même situation. donc cela veut dire que on a une situation $s_a$ qui est la même que $s_b$. Et entre $s_a$ et $s_b$ un certain nombre de situations.
      %
      %   tous les coups entre $s_a$ et $s_b$ s'annulent deux à deux.
      %
      %   supposons que à $s_{a+1}$ $j_max$ joue un coup améliorant sur l'argument $a$. SPDG supposons que la valeur monte.
      %
      %   donc à un état $s_{a+i}$ $j_max$ voudra descendre la valeur et annulera son coup sur l'argument $a$.
      %
      %   Cette situation peut arriver seulement si la valeur arrive au dessus de $j_max$.
      %
      %   à l'état $s_{a+2}$ si l'agent $j_min$ fait diminuer la valeur alors $j_max$ ne voudra pas changer la valeur sur $a$. Si il fait augmenter la valeur alors nécessairement cette valeur va se rapprocher de $j_max$. Il ne voudra donc pas changer d'avis.
      %
      %   à l'état $s_{a+3}$ l'agent $j_max$ pourra faire augmenter la valeur ou la faire descendre. mais sans changer son vote sur $a$. Donc lors du vote $s_{a+4}$ il y a aura 3 votes à annuler.
      %   Si il fait augmenter la valeur alors $j_min$ pourra annuler son vote. Si il l'a fait diminuer alors la valeur va se rapprocher de $j_min$ donc $j_min$ ne pourra pas changer son vote.
      %
      %   Supposons qu'il annule. Donc la valeur va diminuer.
      %
      %   entre $s_a$ et $s_b$ toutes les situations rapprochent soit le joueur $max$ soit le joueur $min$. Aucune situation peut etre pire que $s_a$.
      % \end{proof}


      Pour montrer ce théorème on commence par montrer le lemme suivant:
      \begin{definition}
        Soit $AN$ l'ensemble des arguments sur lesquels aucun des deux joueurs ne c'est prononcés.
      \end{definition}

      \begin{lemme}
        \label{lem:question_5}
        A deux joueurs, si la valeur $v_s$ de l'état $s$ est entre $v_{max}$ et $v_{min}$ alors les joueurs vont changer d'avis sur les arguments de $AN$ un nombre limité de fois.
      \end{lemme}
      \begin{proof}
        Soit $s$ un état où la valeur $v_s$ du débat est entre $v_{max}$ et $v_{min}$. Notons que dans ce cas les joueurs sont adversaires.

        Supposons que le joueur $j_{max}$ vote sur l'argument $a \in AN$. On doit prouver qu'il voudra changer son vote un nombre limité de fois.
        Le vote sur $a$ augmente la valeur du débat car $v_s$ est inférieure à $v_{max}$. A l'état $s+1$, $j_{min}$ doit faire baisser la valeur du jeu et il peut obtenir une valeur au moins égale à $v_s$ en jouant contre $j_{max}$ sur l'argument $a$.

        Supposons que plus tard dans le jeu, à l'état $s'$, $j_{max}$ veut changer son vote sur $a$. Donc il veut que la valeur du jeu diminue. La seule possibilité pour cela est qu'il est changé d'avis sur un argument qui n'appartient pas à $AN$ (sinon $j_{min}$ pourrait contrer son vote) et que la valeur est encore au dessus de $v_{max}$ lorsque c'est à lui de jouer.

        Donc à l'état $s'+1$ la valeur est plus faible qu'a l'état $s'$, donc $j_{min}$ ne pourra toujours pas jouer. Cette remarque est vrai pour tous les états après l'état $s'$, donc $j_{max}$ joue tout seul et donc il peut changer qu'un nombre de fois limité sur chacun des arguments.
      \end{proof}

      prouvons maintenant le théorème \ref{thm:question_5}
      \begin{proof}
        Comme le nombre d'états est fini, on a juste à montrer qu'il n'y a pas de cycle. Pour cela on va montrer que chaque joueur peut changer d'avis qu'un nombre limité de fois sur chacun des arguments.

        On peut séparé le jeu en trois cas :
        \begin{enumerate}
          \item La valeur $v_0$ est entre  $v_{max}$ et $v_{min}$.
          \item La valeur $v_0$ est supérieure (resp. inférieure) à $v_{max}$ (resp. $v_{min}$) et la valeur arrive entre $v_{max}$ et $v_{min}$ à l'état $s$.
          \item Durant tout le jeu, la valeur du jeu n'est jamais entre $v_{max}$ et $v_{min}$
        \end{enumerate}

        \paragraph{Cas 1}
        Tous les arguments sont dans $AN$ donc par le lemme \ref{lem:question_5} on peut conclure que le nombre de fois que chaque player change d'avis est fini. (En réalité, il ne changent jamais leurs avis).

        \paragraph{Cas 2}
        Soit $s$ tel que $v_{min} \leq v_s \leq v_{max}$ et que $s$ soit le premier état avec cette propriété. Grâce au lemme \ref{lem:question_5} On a juste à prouver que les joueurs changent un nombre fini de fois d'avis sur les arguments n'appartenant pas à $AN$.

        SPDG, supposons que $v_0 > v_{max}$. Donc les joueurs ont joué en collaboration (pour faire diminuer la valeur) sur les arguments jouer avant l'état $s$.

        Donc sur tous les arguments le joueur $j_{min}$ a voté pour faire descendre la valeur ou alors il n'a pas voté.

        Supposons que la valeur $v_t$ d'un état $t$ se trouve inférieure à $v_{min}$ et que c'est à $v_{min}$ de jouer.
        Tous les coups réalisés sur les arguments de $AN$ pouvant être contrés par $j_{max}$, cette situation peut arriver uniquement lors d'un vote sur un argument n'appartenant pas à $AN$.
        $j_{max}$ ne peux pas engendrer un tel coups car ce n'est pas un coups améliorant.
        Si $j_{min}$ vote sur l'argument $a \not\in AN$ sur lequel il n'a jamais voter.
        Alors $j_{max}$ peut retirer sont vote sur $a$ et faire remonter la valeur à la valeur précédente.
        Enfin $j_{min}$ ne peux changer aucun de ses votes sur les arguments n'appartenant pas à $AN$ car ils font déjà descendre la valeur.
        Donc cette situation ne peut pas arriver.

        Donc à partir d'un état seul $j_{max}$ peut jouer et donc on converge.


        \paragraph{case 3}
          ils y a deux possibilités :

          \begin{enumerate}
            \item la valeur reste toujours au dessus (resp. en dessous) de $v_{max}$ (resp. $v_{min}$).
              Donc les joueurs jouent en collaboration pour faire descendre la valeur et donc on converge.

            \item La valeur oscille entre : supérieur à $v_{max}$ et inférieur à $v_{min}$ et donc les joueurs vont opposé leurs votes et on arrivera également à une situation stable.
          \end{enumerate}
      \end{proof}

\color{black}
  \section{Questions}
  \color{red}
  Question sur les fonction potentielles, avec les conditions qu'elle doit respecter.
\color{black}
  \section{Ce qu'il faut que je fasse}

    \begin{itemize}
      \color{blue}
      \item regarder, sur le petit jeu à 2 joueurs, si entre chaque valeur possible pour 1 joueur il y a toujours une valeur du jeu. Si non alors il n'existe pas forcement d'équilibre entre $j_{min}$ et $j_{max}$. Je peut également tester si c'est un équilibre.
      \item coder la detection de situations identiques afin de savoir si en better response et/ou en random on repasse plusieurs fois par une situation connu.
      \item Continuer la recherche d'une fonction potentielle?
    \end{itemize}

\end{document}
